\chapter{Related Work}
\label{cha:related-work}

This thesis presents a specialization of existing techniques, along with targeted improvements for our specific use case.
For example, our Context Gathering phase, as it represents a specialized implementation of RAG (Retrieval Augmented Generation).
As Y.Gao et al. \cite{gao2023retrieval} demonstrate, numerous RAG techniques exist; however, our version is optimized for continuous support chat environments by filtering out outdated, resolved, and irrelevant data to focus exclusively on the customer's current needs.
Another example is our application of the \textit{LLM-as-a-Judge} methodology: D. Li et al. \cite{llmasajudge_applications} provide a comprehensive taxonomy of research in this area.
This thesis contributes a practical implementation with evaluation results, featuring specialized evaluation criteria tailored specifically to the e-commerce domain.

In examining similar approaches to LLM deployment in customer service, Wulf et al. \cite{wulf2024exploringpotentiallargelanguage} present a practical implementation for telecommunications support, demonstrating feasibility through prototyping with actual customer data.
Their cognitive task categorization parallels our Context Gathering and Answer Generation phases.
While they focus on technical feasibility and manual validation, our implementation emphasizes autonomous evaluation via LLM-as-a-Judge methodology and business metrics, extending to comprehensive cost analysis and deployment scenarios for e-commerce.
This addresses their identified gap for large-scale validation by providing a framework for model selection based on operational constraints and profit margins.
