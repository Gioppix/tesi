\documentclass[epsfig,a4paper,11pt,titlepage,twoside,openany]{book}
\usepackage{epsfig}
\usepackage{plain}
\usepackage{setspace}
\usepackage[paperheight=29.7cm,paperwidth=21cm,outer=1.5cm,inner=2.5cm,top=2cm,bottom=2cm]{geometry} % per definizione layout
\usepackage{titlesec} % per formato custom dei titoli dei capitoli
\usepackage[]{todonotes}
\usepackage[]{lineno}
\usepackage{hyperref}
\usepackage{cleveref}
\usepackage{enumitem}
\setlist{noitemsep,topsep=0pt}
\linenumbers


\usepackage[utf8]{inputenc} % per UTF-8.

\singlespacing

\usepackage[english]{babel}

\input{hypenations}


% CMDs
\newcommand{\tdin}[1]{\todo[color=blue!30,inline]{TODO: #1}}

\begin{document}

% nessuna numerazione
\pagenumbering{gobble}
\input{first_page}

\clearpage

\thispagestyle{empty}

\begin{center}
    {\bf \Huge Acknowledgements}
\end{center}

\vspace{4cm}

\emph{
    Thanks to:
    Marco Patrignani,
    Alessandro Paladin (if Kuba is included),
    family (?),
    Claude for giving me tier 4 api access allowing me to give them money faster
}

\tdin{expand}

\clearpage
\pagestyle{plain} % nessuna intestazione e pie pagina con numero al centro

% inizio numerazione pagine in numeri arabi
\mainmatter

% indice
\tableofcontents
\clearpage

% gruppo per definizone di successione capitoli senza interruzione di pagina
\begingroup
% nessuna interruzione di pagina tra capitoli
% ridefinizione dei comandi di clear page
\renewcommand{\cleardoublepage}{}
\renewcommand{\clearpage}{}
% redefinizione del formato del titolo del capitolo
% da formato
%   Capitolo X
%   Titolo capitolo
% a formato
%   X   Titolo capitolo

\titleformat{\chapter}
{\normalfont\Huge\bfseries}{\thechapter}{1em}{}

\titlespacing*{\chapter}{0pt}{0.59in}{0.02in}
\titlespacing*{\section}{0pt}{0.20in}{0.02in}
\titlespacing*{\subsection}{0pt}{0.10in}{0.02in}

\chapter{Introduction}
\label{cha:introduction}

% \section{Abstract}

% Context

E-commerce businesses rely heavily on email marketing to drive revenue growth, but emails have become increasingly saturated and suffer from declining engagement rates.
This has created a pressing need for these businesses to find more effective communication channels to reach their customers.
The messaging platform \textit{WhatsApp} recently released APIs to interact with the platform in a systematic manner, notably the ability to automatically send messages.
This enables e-commerce businesses to leverage WhatsApp as an alternative communication channel that offers significantly higher open and response rates than email while supporting the same marketing capabilities including automated campaigns, customer segmentation, and targeted messaging.
E-commerce businesses can easily establish automated messaging workflows, and customers frequently engage with these communications by responding to the messages.

% Problem

However, this increased engagement presents both opportunities and challenges: e-commerce businesses must now manage a substantially larger volume of customer inquiries, many of which are tedious and do not require specialized knowledge nor complex actions.
WhatsApp has platform-specific limitations that worsen this challenge: it enforces strict restrictions on message responses, particularly limiting businesses from responding to customer messages after a 24-hour window has elapsed since the last customer-initiated message \cite{wa_policy}.

% Solution

This thesis presents the implementation of an LLM-powered assistant for e-commerce customer service.
The assistant autonomously guides customers through the purchase process by suggesting relevant products, answering order inquiries about contents and shipping status, and providing business-specific information.
The system also aids human intervention by prioritizing conversations and categorizing conversations to enable different departments to operate independently and efficiently.
The architecture chosen for the assistant employs a 2-phase approach comprising \textit{Context Gathering} and \textit{Answer Generation}, where relevant context is retrieved between phases to narrow down the information provided to the final response generation.
This architectural necessity enables the system to efficiently handle large product catalogs and order histories by first identifying relevant information and then generating appropriate responses based on that context.
Since implementing such a system can be accomplished through various platforms, including no-code solutions, the primary contribution of this thesis lies in the validation.

% Validation

In particular, we focus on developing a systematic process that can automatically and objectively evaluate its performance, in function of the models chosen to complete the tasks. Evaluation covers five key dimensions, listed in order of priority:
\begin{itemize}
    \item Hallucinations: instances where the model generates factually incorrect or fabricated information not supported by the provided context
    \item Accuracy: the ability to correctly utilize provided contextual information and adhere to given instructions
    \item Cost: computational expense measured in dollars per response, accounting for all model inference calls
    \item Speed: response generation latency measured in seconds, including only model processing time
\end{itemize}
We validate each of the two phases separately, leveraging their independent operational capability (by providing standardized, synthetically generated context to the \textit{Answer Generation} phase).
For each phase, the selected model processes multiple synthetic chat conversations, and its outputs are evaluated by another LLM operating as a judge (LLM-as-a-Judge), which has been manually fine-tuned to align with human preferences.
The use of a secondary LLM for evaluation is necessary to assess accuracy, which cannot be measured through simple metrics like cost or speed.
\tdin{Aggiungere risultati}

\section{Outline}

\tdin{check that content is correct}

We begin by establishing the theoretical foundations in \cref{cha:background}, examining Large Language Models, the specific contextual requirements of e-commerce customer service, Retrieval Augmented Generation techniques, and vector database technologies that enable semantic search capabilities.

\Cref{cha:impl} presents the core implementation of the assistant, detailing our 2-phase architecture comprising Context Gathering and Answer Generation phases. The Context Gathering phase uses RAG techniques and smaller models to efficiently select relevant information from product catalogs and order histories, while the Answer Generation phase covers output formatting strategies, available actions like message sending, conversation categorization and priority assessment.

\Cref{cha:validation} focuses on the systematic evaluation process, describing how we automatically and objectively assess the system's performance across different model configurations, including the development of synthetic test datasets and the LLM-as-a-Judge evaluation methodology.

Finally, \cref{cha:conclusion} synthesizes our findings regarding the evaluation framework and its results, complete with trade-offs between accuracy, cost, speed, and environmental impact in different LLM models.

% \Cref{cha:related-work} positions our work within the existing literature on automated customer service systems and LLM applications in commercial contexts.


\newpage
\chapter{Background}
\label{cha:background}

This chapter establishes the theoretical foundations necessary for understanding the implementation and evaluation of an LLM-powered customer service assistant for e-commerce applications. We begin by examining the capabilities and limitations of Large Language Models in practical applications, followed by an analysis of the specific contextual requirements for e-commerce customer service automation. We then explore Retrieval Augmented Generation (RAG) techniques that enable LLMs to access external knowledge sources, and conclude with an overview of vector database technologies that provide the semantic search capabilities essential for efficient information retrieval in our system architecture.
\tdin{Check updates}

\section{Large Language Models}

Large Language Models (LLMs) are neural network architectures trained to understand and generate natural language responses. These models demonstrate strong performance across general knowledge tasks but require domain-specific context to be effectively applied in specialized applications such as e-commerce customer service.

The fundamental limitation of LLMs for practical applications lies in their training data cutoff and lack of access to real-time, business-specific information. While these models possess broad linguistic capabilities and general world knowledge, they cannot access current product catalogs, order statuses, or company-specific policies without external information retrieval mechanisms.

\section{Domain-Specific Context in the E-commerce Niche}

This thesis focuses on automating customer service responses that primarily involve retrieving and presenting existing information rather than executing transactional actions. The automated assistant is designed to handle informational queries by processing and presenting data already stored within the e-commerce system, emphasizing information delivery over system modifications or direct transaction processing.
In particular, three main categories of information are required to effectively respond to customer inquiries:
\begin{itemize}
    \item Order history information for questions regarding order status and contents
    \item Available product information to guide customers through the purchase process and provide relevant product recommendations
    \item E-commerce business rules including accepted payment methods, available delivery options, and courier information
\end{itemize}

\section{Retrieval Augmented Generation}
\label{sec:rag}

RAG is a technique that combines information retrieval with language generation to provide more accurate and contextual responses.
The implementation of RAG follows a straightforward process: when a user submits a query, the system first converts the query into a vector representation and searches a vector database for semantically similar content.
The most relevant information is then retrieved and provided as context to the LLM, which generates a response that incorporates this retrieved knowledge alongside its pre-trained capabilities.
This approach allows LLMs to access up-to-date, domain-specific information that wasn't present in their original training data.

\section{Vector Databases}
\label{sec:vecdb}

Vector databases are specialized storage systems designed to efficiently store and retrieve high-dimensional vector representations of data; in our case, for example, text embeddings generated by machine learning models.
They enable semantic search capabilities by measuring similarity between vectors, allowing systems to find contextually relevant information even when exact keyword matches don't exist.

\section{Prompt Engineering}
\label{sec:prompt-engineering}

Prompt engineering is the practice of designing and optimizing input prompts to guide LLM behavior and improve response quality.
Effective prompts provide clear instructions, appropriate context, and formatting guidelines that help the model generate more accurate and relevant responses.
In customer service applications, well-crafted prompts can significantly improve the consistency and helpfulness of automated responses by establishing the assistant's role, communication style, and response structure.


\newpage
\chapter{Implementation}
\label{cha:impl}

This chapter presents the core implementation of our LLM-powered assistant for e-commerce customer service, detailing the 2-phase architecture designed to address computational cost scalability challenges while maintaining high response quality.
We examine the overall architecture (\cref{sec:architecture}), discuss output format considerations (\cref{sec:output_format}) across different LLM providers, and then detail the Context Gathering (\cref{sec:context-gathering}), Intermediate (\cref{sec:intermediate}), and Answer Generation (\cref{sec:answer-generation}) phases.
Finally, we describe the reliability mechanisms (\cref{sec:reliability}) implemented to handle model inconsistencies.

\section{The 2-Phase Architecture}
\label{sec:architecture}

An analysis of the pricing structures for Large Language Models used in this thesis \cite{gemini_pricing, anthropic_pricing, openai_pricing} demonstrates that computational costs increase proportionally with the volume of context tokens processed by the model.
This observation highlighted a fundamental scalability challenge: most e-commerce platforms maintain catalogs containing hundreds to thousands of products and variations, making it impractical to include all available information in every model inference.

To address this limitation we implemented a 2-phase architecture, where the Context Gathering Phase receives the conversation history as input and decides which contextual elements should be included in the subsequent Answer Generation Phase.
This approach represents a more sophisticated form of RAG (\cref{sec:rag}) that ensures older or resolved conversations do not degrade output quality by selectively retrieving only the most relevant contextual information.
This approach offers significant scalability advantages and provides a framework that can be extended to incorporate additional context categories as system requirements evolve.

\section{Output Format Considerations}
\label{sec:output_format}

A critical design consideration for the implementation involves the output structure format.
Since the output of the LLM undergoes automatic parsing, it must maintain consistent structure with predefined fields that cannot be left to the discretion of the model.

Various LLM providers offer different levels of output formatting control.
Some models provide a \textit{structured output} mode, which accepts a JSON schema and constrains the response to conform to the specified structure.
Other models offer only a \textit{JSON mode} that enforces JSON formatting without a predefined schema structure.
Some models do not support structured output formatting at all, requiring reliance solely on prompt engineering techniques.

For this implementation, we selected JSON as the output format over alternatives such as XML due to its widespread support across different model providers.
Models that do not support JSON mode (notably those from Anthropic) or those with intermediate capabilities (such as older OpenAI models) require additional prompt engineering.
These models receive a TypeScript representation of the desired output format within their prompt, which provides a more compact specification than JSON Schema while maintaining sufficient flexibility for the requirements of the system.
Since these models are trained to output code formatted as Markdown, they typically wrap JSON responses in code blocks, requiring the removal of Markdown formatting from the beginning and end of the response.
While prompt engineering techniques were explored to eliminate this behavior during development, they proved inconsistent across different queries and contexts, whereas simple string trimming provides a reliable solution for extracting the JSON content.

\section{System Architecture Details}

The following subsections detail the specific implementation of each step in our 2-phase architecture, explaining how contextual information flows through the system and how each component contributes to generating appropriate customer service responses.

\subsection{Context Gathering Phase}
\label{sec:context-gathering}

The Context Gathering Phase utilizes a language model that has been specifically configured to analyze conversational context and determine the relevant information required for generating an appropriate response.
The output structure of this phase consists of two primary components: a \textit{boolean} value that indicates whether order history should be incorporated into the context, and a nullable \textit{string} parameter that contains a search query for product retrieval.
When generating such queries, the language model is instructed to formulate queries optimized for vector database similarity search operations (\cref{sec:vecdb}).

This phase could also be used to synthesize the chat history to optimize Phase 2 by providing a condensed summary of the conversation; however, in this thesis we focus on shorter conversations and maximum accuracy, prioritizing the preservation of complete conversational context over computational efficiency gains from summarization.

\subsection{Intermediate Phase}
\label{sec:intermediate}

When a query is generated during the Context Gathering phase, the same embedding model that was employed for the product catalog indexing converts it into an embedding, ensuring consistency in the vector space representation.
A similarity search is then performed against the product catalog embeddings, with the top-N most relevant products being selected based on cosine similarity scores (a threshold is also used to avoid including unrelated information).
These selected products, along with order information when requested, are subsequently formatted and passed to the Answer Generation Phase.

\subsection{Answer Generation Phase}
\label{sec:answer-generation}

The Answer Generation Phase receives the contextual information assembled from the previous phases along with the complete conversation history.
In this phase, \cref{sec:prompt-engineering} has a much greater impact on the quality and effectiveness of the generated responses than in the first phase.
The output structure consists of multiple components:
\begin{itemize}
    \item A \textit{Priority} enumeration that indicates the urgency level of the conversation (ranging from no urgency to urgent human intervention required)
    \item A \textit{Mood} enumeration that captures the customer's emotional state
    \item A list of \textit{categories} that tag the conversation with relevant business classifications
    \item An \textit{actions} array (that might be empty) containing the actions the model choose to take and their parameters
\end{itemize}
The use of enumerations for \textit{Priority} and \textit{Mood} provides significant advantages as they effectively constrain the output to predefined, valid options, ensuring consistency and reliability in the structured responses while preventing the generation of unexpected or malformed values.

The actions array represents a flexible and scalable architectural design that enables future system expansion.
Currently, the system implements a single action type for sending messages; however, the framework can be trivially extended to incorporate additional action types such as tag modification or checkout creation without requiring fundamental architectural changes.
This structured approach also enhances the awareness of the model regarding its operational context and decision-making process.
During the development phase, empirical observations revealed that models would sometimes misuse an old \textit{message} field for commentary or justification purposes rather than customer-facing communication, a behavior that the structured action framework effectively eliminates.

\section{Model Reliability and Error Mitigation}
\label{sec:reliability}

Both model calls in the two-phase architecture implement a retry mechanism with a maximum limit of three attempts to mitigate potential failures.
The vast majority of failures encountered during system operation occur during the deserialization process, specifically in instances where language models fail to adhere to the specified output structure formats.

To ensure fair evaluation across different models, the cumulative costs and execution times across multiple retry attempts are recorded and summed for each query.
This approach provides an accurate representation of the total computational overhead required by models that demonstrate varying levels of consistency in output format adherence.

Empirical testing conducted during the development phase revealed that flagship models achieved error rates of less than 1\% in output format compliance.
Models equipped with structured output capabilities demonstrated effectively zero error rates, highlighting the significant reliability improvements offered by constrained generation features.
While this implementation detail will not be examined in depth in the validation chapter, its effects on system performance are reflected in the measured costs and execution times presented in the results.


\newpage
\chapter{Validation}
\label{cha:validation}

This chapter presents the systematic evaluation of our LLM-powered customer service assistant using multiple models from major providers across both phases of our architecture.
We employ an automated evaluation framework using synthetic chat conversations and LLM-as-a-Judge methodology to assess model performance across accuracy, cost, and speed dimensions.
The evaluation reveals distinct performance characteristics between the simpler Context Gathering phase and the more complex Answer Generation phase, providing insights for production deployment decisions.

\section{Test Dataset}

The test dataset comprises 80 synthetically generated chat conversations.
To ensure fairness and consistency in the evaluation process, all models are tested using this identical dataset.

The size of 80 conversations was selected based on two key considerations: it provides sufficient diversity to encompass a wide range of conversational scenarios and enables thorough testing of model consistency, while simultaneously maintaining manageable computational costs and reasonable testing duration.

\section{Models Under Evaluation}

We tested major LLM providers using their latest available models.
For models that supported both configurations, we evaluated them with built-in Chain of Thoughts (CoT) reasoning enabled and disabled (see \cref{sec:thinking}).
All models were configured with virtually unlimited output token limits, with the primary constraint being the thinking effort or thinking budget for models that support reasoning capabilities.
Where possible, we tuned this thinking parameter to even out cost and response speed.

However, not all available models were included in our evaluation due to cost constraints. Several high-end models were excluded:
\begin{itemize}
    \item \textbf{Anthropic}: Claude Opus 4.0 \cite{anthropic_pricing}
    \item \textbf{OpenAI}: GPT 4.5, o1 Pro, o3 Pro \cite{openai_pricing}
\end{itemize}

\section{Automated Evaluation Using LLM-as-a-Judge}

Given the extensive test dataset combined with multiple models under evaluation, the analysis process generates thousands of responses that require assessment.
Manual evaluation of this volume of data is impractical due to resource constraints and time limitations.

Large Language Models serve as an effective solution for automated evaluation in this context.
The approach requires a single comprehensive prompt engineering phase to establish evaluation criteria that align with human judgment standards, after which the model can consistently apply these criteria across all responses (as J. Gu et al. \cite{llmasajudge} presents).

This methodology is grounded in the principle that evaluation tasks are generally less complex than generation tasks, as judging the quality of an existing response requires analyzing and assessing predetermined criteria rather than synthesizing new information from scratch.
The evaluator model can focus exclusively on determining whether specific quality standards are met, without needing to balance creativity, factual accuracy, and contextual appropriateness simultaneously as required in generation tasks.
Furthermore, this evaluation approach creates an extra layer of review, allowing the LLM to potentially identify errors that might have been missed, even when the same model evaluates its own generated responses.

During the development phase, a reasoning field was incorporated into the evaluation framework to provide transparency regarding the judge model's decision-making process, enabling iterative refinement of the evaluation criteria.

We selected Claude Sonnet 4.0 with built-in CoT enabled as our judge model because it demonstrated the best alignment with manual evaluation standards while maintaining cost-effectiveness.
To validate this choice, we tested alternative models in the judge role and confirmed that the overall performance rankings remained consistent across different judges, with only minor variations comparable to those observed between human evaluators.
Interestingly, we observed that judging models tended to assign slightly higher scores when evaluating responses generated by identical models. This phenomenon does not represent bias, as the judge model had no visibility into which model generated each response under evaluation.
We attribute this alignment to consistent interpretation of instructions between identical models (for instance, how each model defines concepts such as hallucination).
While our evaluation prompts are comprehensive and detailed, some degree of interpretive variation is inherent in any evaluation process.

\section{Phase 1 Evaluation}

The accuracy evaluation for Phase 1 is based on two dimensions:
\begin{itemize}
    \item \textbf{Products Query Accuracy}: This measures whether a query is accurate. A query is considered accurate if it is absent when not needed, or if it is present when needed and correctly captures what the customer requires.
    \item \textbf{Include Orders Accuracy}: This measures the accuracy of including past orders in the context.
\end{itemize}
To monitor performance, another three key metrics are introduced:
\begin{itemize}
    \item \textbf{Failure Rate}: The percentage of cases where models fail to generate valid JSON output for all 3 attempts.
    \item \textbf{Cost}: The computational expense measured in dollars per response, including only model inference calls.
    \item \textbf{Speed}: The response generation time measured in seconds, which includes only the model processing time.
\end{itemize}

\subsection{Results for Phase 1}

The results show that Phase 1 evaluation is a relatively simple task.
All models performed well across both evaluation dimensions, with consistently low failure rates.
The only notable exception was Claude Haiku 3.5, which struggled to produce consistent JSON output.
The remaining models either had access to JSON mode functionality (detailed in \cref{sec:output_format}) or, in the case of more advanced models, were capable enough to complete the test without requiring this feature.
These results indicate that multiple models would be suitable for production use, allowing organizations to prioritize either response speed or cost efficiency based on their specific requirements.

\begin{table}[H]
    \centering
    \footnotesize
    \begin{tabularx}{\textwidth}{Xp{2.5cm}p{2.2cm}p{1.5cm}p{1.8cm}p{2cm}}
        \toprule
        Model                       & Products Query \newline Accuracy (\%) & Include Orders \newline Accuracy (\%) & Failure \newline Rate (\%) & Average \newline Duration (s) & Average \newline Cost (\$ cents) \\
        \midrule
        \rowcolor[gray]{0.9}
        Claude Haiku 3.5            & 88.2                                  & 100.0                                 & 5.0                        & 2.764                         & 0.1006                           \\
        Claude Sonnet 4.0\textbf{*} & 97.5                                  & 98.8                                  & 0.0                        & 9.567                         & 0.6613                           \\
        \rowcolor[gray]{0.9}
        Claude Sonnet 4.0           & 95.0                                  & 98.8                                  & 0.0                        & 2.896                         & 0.2050                           \\
        OpenAI o3-mini\textbf{*}    & 96.2                                  & 100.0                                 & 0.0                        & 4.945                         & 0.1142                           \\
        \rowcolor[gray]{0.9}
        OpenAI o4-mini\textbf{*}    & 95.0                                  & 100.0                                 & 0.0                        & 3.283                         & 0.0518                           \\
        OpenAI o1\textbf{*}         & 97.5                                  & 98.8                                  & 0.0                        & 5.302                         & 2.2228                           \\
        \rowcolor[gray]{0.9}
        OpenAI GPT-4.1              & 98.8                                  & 100.0                                 & 0.0                        & 1.345                         & 0.0946                           \\
        Gemini 2.5 Pro\textbf{*}    & 96.2                                  & 98.8                                  & 0.0                        & 6.859                         & 0.5644                           \\
        \rowcolor[gray]{0.9}
        Gemini 2.5 Flash\textbf{*}  & 95.0                                  & 97.5                                  & 0.0                        & 3.318                         & 0.1224                           \\
        Gemini 2.5 Flash            & 93.8                                  & 97.5                                  & 0.0                        & 0.996                         & 0.0202                           \\
        \rowcolor[gray]{0.9}
        Gemini 2.5 FL\textbf{*}     & 96.2                                  & 97.5                                  & 0.0                        & 1.535                         & 0.0240                           \\
        Gemini 2.5 FL               & 95.0                                  & 100.0                                 & 0.0                        & 0.641                         & 0.0052                           \\
        \bottomrule
    \end{tabularx}

    \begin{itemize}
        \footnotesize
        \item Models marked with \textbf{*} had built-in CoT enabled.
    \end{itemize}
    \caption{Results of different language models in Phase 1 evaluation}
    \label{tab:phase1_results}
\end{table}

\section{Phase 2 Evaluation}

The accuracy evaluation for Phase 2 is based on five key dimensions:
\begin{itemize}
    \item \textbf{Response Rate}: Percentage of times the model chooses to provide a response. A 100\% response rate is not the target, as the test dataset deliberately includes scenarios where the customer service assistant should abstain from responding. Higher response rates are only beneficial when accompanied by high accuracy and low hallucination rates.
    \item \textbf{Message Accuracy}: This evaluates whether the actions appropriately address the customer's needs based on the conversation context. This metric is calculated exclusively for cases where a message is present in the output.
    \item \textbf{Absence of Hallucinations}: This measures whether the model generates factually incorrect or made-up information that is not supported by the provided context. For this evaluation, we only consider the message content and exclude the categories. This metric is calculated exclusively for cases where a message is present in the output.
    \item \textbf{Priority Accuracy}: This measures whether the model correctly assigns priority levels to chats.
    \item \textbf{Categories Accuracy}: This evaluates whether the model accurately categorizes the chats.
\end{itemize}
Additionally, as in Phase 1, we include three performance metrics:
\begin{itemize}
    \item \textbf{Failure Rate}
    \item \textbf{Cost}
    \item \textbf{Speed}
\end{itemize}

\subsection{Results for Phase 2}

Phase 2 presents a significantly more complex evaluation scenario, and the results demonstrate that models begin to encounter greater challenges with this increased complexity.
The evaluation revealed a notable increase in hallucinations across most models.

Claude Sonnet 4.0 without thinking capabilities, despite demonstrating strong reasoning abilities, struggled to maintain consistent JSON output formatting.
Claude Sonnet 4.0 with thinking enabled achieved the highest performance in terms of accuracy and minimized hallucinations.
However, this model exhibited notably conservative behavior, as evidenced by its lowest response rate among all tested models; additionally, this configuration was the slowest performing model and ranked as the second most expensive option.
While this performance might initially appear to be influenced by the fact that Claude Sonnet 4.0 also served as the judge model, we confirmed this result by conducting additional evaluations using a Gemini model as the judge.

An unexpected finding emerged regarding the Gemini model variants: Gemini 2.5 Flash outperformed Gemini 2.5 Pro while operating approximately 4 times faster and 8 times more cost-effectively.
To verify this counterintuitive result, we conducted manual verification of the judge's assessments and confirmed that the smaller model indeed delivered superior performance (refer to \cref{sec:pro_vs_flash} for a detailed example).

Unlike Phase 1, selecting an appropriate model for production deployment in Phase 2 scenarios requires careful consideration of trade-offs between quality, speed, and cost efficiency.

\begin{table}[h]
    \centering
    \footnotesize
    \begin{tabularx}{\textwidth}{Xp{1.1cm}p{1.25cm}p{1.6cm}p{1.3cm}p{1.3cm}p{1.3cm}p{1.3cm}p{1.3cm}}
        \toprule
        Model                            & Response \newline Rate (\%) & Response \newline Accurate (\%) & Hallucinated Info (\%) & Priority Accurate (\%) & Categories Accurate (\%) & Failure Rate (\%) & Average \newline Duration (s) & Average Cost \newline (\$ cents) \\
        \midrule
        \rowcolor[gray]{0.9}
        Claude Haiku 3.5                 & 95.0                        & 76.3                            & 25.0                   & 85.5                   & 100.0                    & 5.0               & 4.812                         & 0.3717                           \\
        Claude Sonnet 4.0\textbf{*}      & 68.8                        & 96.4                            & 7.3                    & 86.2                   & 100.0                    & 0.0               & 23.963                        & 1.9078                           \\
        \rowcolor[gray]{0.9}
        Claude Sonnet 4.0                & 80.0                        & 90.6                            & 14.1                   & 92.4                   & 100.0                    & 1.2               & 11.402                        & 0.8094                           \\
        OpenAI o3-mini\textbf{*}         & 88.8                        & 81.7                            & 12.7                   & 83.8                   & 100.0                    & 0.0               & 5.859                         & 0.3231                           \\
        \rowcolor[gray]{0.9}
        OpenAI o4-mini\textbf{*}         & 83.8                        & 83.6                            & 34.3                   & 81.2                   & 100.0                    & 0.0               & 3.987                         & 0.1536                           \\
        OpenAI o1\textbf{*}              & 82.5                        & 90.9                            & 10.6                   & 82.5                   & 100.0                    & 0.0               & 6.506                         & 5.5855                           \\
        \rowcolor[gray]{0.9}
        OpenAI GPT-4.1                   & 76.3                        & 91.8                            & 13.1                   & 86.2                   & 100.0                    & 0.0               & 2.689                         & 0.4141                           \\
        Gemini 2.5 Pro\textbf{*}         & 75.0                        & 88.3                            & 10.0                   & 83.8                   & 100.0                    & 0.0               & 15.834                        & 1.5860                           \\
        \rowcolor[gray]{0.9}
        Gemini 2.5 Flash\textbf{*}       & 73.8                        & 89.8                            & 8.5                    & 87.5                   & 100.0                    & 0.0               & 4.239                         & 0.2153                           \\
        Gemini 2.5 Flash                 & 76.3                        & 86.9                            & 19.7                   & 85.0                   & 100.0                    & 0.0               & 1.242                         & 0.0752                           \\
        \rowcolor[gray]{0.9}
        Gemini 2.5 \textit{FL}\textbf{*} & 68.8                        & 89.1                            & 9.1                    & 83.8                   & 95.0                     & 0.0               & 2.662                         & 0.0586                           \\
        Gemini 2.5 \textit{FL}           & 72.5                        & 91.4                            & 19.0                   & 86.2                   & 78.8                     & 0.0               & 1.900                         & 0.0425                           \\
        \bottomrule
    \end{tabularx}

    \begin{itemize}
        \footnotesize
        \item \textit{FL} stands for Flash Lite
        \item Models marked with \textbf{*} had built-in CoT enabled.
    \end{itemize}
    \caption{Results of different language models in Phase 2 evaluation}
    \label{tab:phase2_results}
\end{table}


\newpage
\chapter{Related Work}
\label{cha:related-work}

This thesis presents a specialization of existing techniques, along with targeted improvements for our specific use case.
For example, our Context Gathering phase, as it represents a specialized implementation of RAG (Retrieval Augmented Generation).
As Y.Gao et al. \cite{gao2023retrieval} demonstrate, numerous RAG techniques exist; however, our version is optimized for continuous support chat environments by filtering out outdated, resolved, and irrelevant data to focus exclusively on the customer's current needs.
Another example is our application of the \textit{LLM-as-a-Judge} methodology: D. Li et al. \cite{llmasajudge_applications} provide a comprehensive taxonomy of research in this area.
This thesis contributes a practical implementation with evaluation results, featuring specialized evaluation criteria tailored specifically to the e-commerce domain.

In examining similar approaches to LLM deployment in customer service, Wulf et al. \cite{wulf2024exploringpotentiallargelanguage} present a practical implementation for telecommunications support, demonstrating feasibility through prototyping with actual customer data.
Their cognitive task categorization parallels our Context Gathering and Answer Generation phases.
While they focus on technical feasibility and manual validation, our implementation emphasizes autonomous evaluation via LLM-as-a-Judge methodology and business metrics, extending to comprehensive cost analysis and deployment scenarios for e-commerce.
This addresses their identified gap for large-scale validation by providing a framework for model selection based on operational constraints and profit margins.


\newpage
\chapter{Conclusion}
\label{cha:conclusion}

This thesis demonstrates the implementation of an LLM-powered customer service assistant for e-commerce applications.
We consider the implementation successful as it reached our targets of making multiple models achieve high accuracy levels across both phases while keeping average costs and response times to acceptable levels.

The automated evaluation framework using LLM-as-a-Judge proved effective for systematic model comparison.
We validated the judge's assessments through comparison with human judgment and found high agreement rates, confirming the reliability of our evaluation methodology.
This framework provides a reproducible testing approach that can be used to evaluate new models as they become available, enabling ongoing comparison of model performance in customer service applications.

\section{Future Work}

The system can be extended with additional context and actions to build a more comprehensive assistant.
Some examples include:
\newline
\textbf{Input Context Expansions}:
\begin{itemize}
    \item Current promotional campaigns and discount codes
    \item Product reviews and ratings from other customers
    \item Seasonal availability and restocking schedules
    \item Customer loyalty program status and available rewards
\end{itemize}
\textbf{Output Action Expansions}:
\begin{itemize}
    \item Creating and updating support tickets with appropriate priority levels
    \item Generating personalized discount codes for specific customers
    \item Initiating refund processes and return merchandise authorizations
    \item Updating customer preferences and communication settings
    \item Adding products to wish lists or shopping carts
    \item Subscribing customers to product availability notifications
    \item Generating shipping labels and return instructions
\end{itemize}
It would be then interesting to see the results from the evaluation to check how models compare in more and more complex scenarios.


\endgroup

% bibliografia in formato bibtex
%
% aggiunta del capitolo nell'indice
\addcontentsline{toc}{chapter}{Bibliography}
% stile con ordinamento alfabetico in funzione degli autori
\bibliographystyle{plain}
\bibliography{biblio}
%%%%%%%%%%%%%%%%%%%%%%%%%%%%%%%%%%%%%%%%%%%%%%%%%%%%%%%%%%%%%%%%%%%%%%%%%%
%%%%%%%%%%%%%%%%%%%%%%%%%%%%%%%%%%%%%%%%%%%%%%%%%%%%%%%%%%%%%%%%%%%%%%%%%%
%% Nota
%%%%%%%%%%%%%%%%%%%%%%%%%%%%%%%%%%%%%%%%%%%%%%%%%%%%%%%%%%%%%%%%%%%%%%%%%%
%% Nella bibliografia devono essere riportati tutte le fonti consultate
%% per lo svolgimento della tesi. La bibliografia deve essere redatta
%% in ordine alfabetico sul cognome del primo autore.
%%
%% La forma della citazione bibliografica va inserita secondo la fonte utilizzata:
%%
%% LIBRI
%% Cognome e iniziale del nome autore/autori, la data di edizione, titolo, casa editrice, eventuale numero dell’edizione.
%%
%% ARTICOLI DI RIVISTA
%% Cognome e iniziale del nome autore/autori, titolo articolo, titolo rivista, volume, numero, numero di pagine.
%%
%% ARTICOLI DI CONFERENZA
%% Cognome e iniziale del nome autore/autori (anno), titolo articolo, titolo conferenza, luogo della conferenza (città e paese), date della conferenza, numero di pagine.
%%
%% SITOGRAFIA
%% La sitografia contiene un elenco di indirizzi Web consultati e disposti in ordine alfabetico.
%% E’ necessario:
%%   Copiare la URL (l’indirizzo web) specifica della pagina consultata
%%   Se disponibile, indicare il cognome e nome dell’autore, il titolo ed eventuale sottotitolo del testo
%%   Se disponibile, inserire la data di ultima consultazione della risorsa (gg/mm/aaaa).
%%%%%%%%%%%%%%%%%%%%%%%%%%%%%%%%%%%%%%%%%%%%%%%%%%%%%%%%%%%%%%%%%%%%%%%%%%
%%%%%%%%%%%%%%%%%%%%%%%%%%%%%%%%%%%%%%%%%%%%%%%%%%%%%%%%%%%%%%%%%%%%%%%%%%

\titleformat{\chapter}
{\normalfont\Huge\bfseries}{Attachments \thechapter}{1em}{}
\appendix
\chapter{Chat Interaction Examples}

\section{Comparison of Gemini 2.5 Pro and Gemini 2.5 Flash}
\label{sec:pro_vs_flash}

\subsection*{Messages}
\begin{itemize}
    \item \textbf{Customer:} Hello, I need to order some supplies for my office. Can you give me a bulk price quote for 50 reams of your standard printer paper and 200 of your basic black ballpoint pens?
\end{itemize}

\subsection*{Available Products}
\begin{itemize}
    \item \textbf{Standard 8.5x11 Printer Paper}
          \begin{itemize}
              \item Description: 92 brightness, 20 lb. multipurpose printer paper. Ideal for everyday printing. Sold by the ream (500 sheets).
              \item Price: 8.99€
              \item URL: [url omitted]
              \item Variations: 1 Ream - 8.99€
          \end{itemize}

    \item \textbf{Basic Black Ballpoint Pens}
          \begin{itemize}
              \item Description: Reliable and smooth-writing black ballpoint pens with a medium point. Perfect for office use. Sold in boxes of 12.
              \item Price: 5.99€
              \item URL: [url omitted]
              \item Variations: Box of 12 - 5.99€
          \end{itemize}
\end{itemize}

\subsection*{Model Responses}

\begin{table}[H]
    \begin{tabularx}{\textwidth}{|l|X|}
        \hline
        \textbf{Model}   & \textbf{Response}                                                                                                                                                                                                                                                                                                                                                                                                                                                                                                                                                        \\
        \hline
        Gemini 2.5 Pro   & Hi there! Happy to help you with that order. Here are the products you mentioned: - 50 reams of Standard 8.5x11 Printer Paper: [url omitted] - For the pens, they come in boxes of 12, so you'd need 17 boxes for 200 pens: [url omitted] \textcolor{red}{I've passed your message on to our sales team} to see what they can do for you regarding a bulk discount. \textcolor{red}{Someone will be in touch} with a quote soon!                                                                                                                                         \\
        \hline
        Gemini 2.5 Flash & Hello! I can provide you with a quote based on our current pricing for the quantities you're looking for. For 50 reams of our Standard 8.5x11 Printer Paper, the cost would be 449.50 €. You can find more details here: [url omitted]. For 200 Basic Black Ballpoint Pens, since they are sold in boxes of 12, you would need to purchase 17 boxes to get 204 pens. The total cost for 17 boxes would be 101.83 €. You can view the pens here: [url omitted]. The estimated total for your order would be 551.33 €. Please let me know if you have any other questions! \\
        \hline
    \end{tabularx}
\end{table}

\subsection*{Explanation}

The models were given specific instructions not to make claims like those highlighted in red.
This is because the AI models cannot actually pass messages to other teams or guarantee that someone will contact the customer, since it's assumed that human operators respond within the same chat interface.


\end{document}
